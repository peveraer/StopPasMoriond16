\section{Event samples, reconstruction, and selection}
\subsection{Object definition and event preselection}
\label{sec:presel}
The particle-flow (PF) algorithm~\cite{CMS-PAS-PFT-09-001,CMS-PAS-PFT-10-001} is used to reconstruct and 
identify each individual particle with an optimized combination of information from the various elements of the CMS detector.

The majority of data events are selected using trigger that requires large missing transverse momentum. The missing transverse momentum 
vector \ptvecmiss is defined as the projection on the plane perpendicular to the beams of the negative vector sum of the momenta of all 
reconstructed particles in an event. Its magnitude is referred to as \ETmiss. In the trigger selection, $\ETmiss>170\GeV$ is required. 
Use of this trigger alone introduces a small inefficiency with respect to the offline selection of $\MET > 250\GeV$. To recover the efficiency, 
additional triggers are used. These triggers require presence of at least one lepton (electron or muon) with transverse momentum (\pt) 
requirements of greater than $23\GeV$ for the muon and $20\GeV$ for the electron. The combined trigger efficiency, as measured with a data 
sample of events with large scalar sum of jet transverse momenta ($H_{T}$), varies between XX\% and YY\%(XX\% and YY\%) for events containing 
electrons (muons), each with an uncertainty of about XX\%.

The offline selection requires events to have exactly one lepton with $\pt > 20 \GeV$ and $\abs{\eta} < 2.4$ ($\abs{\eta} < 1.442$) 
for muons (electrons). Electrons in the forward region of the detector are not used due to larger backgrounds. 
Electron candidates are reconstructed starting from a cluster of energy deposits in the
electromagnetic calorimeter. The cluster is then matched to a reconstructed track. The electron selection is based on the shower shape,
track-cluster matching, and consistency between the cluster energy and the track momentum~\cite{Khachatryan:2015hwa}. Muon candidates are 
reconstructed by performing a global fit that requires consistent hit patterns in the tracker and the muon system~\cite{MUOART}.

Leptons are required to be isolated from other activity in the event. A measure of lepton isolation is the scalar \pt\ sum
 ($\pt^\text{sum}$) of all PF particles not associated with the lepton within a cone of radius 
$\Delta R \equiv\sqrt{\smash[b]{(\Delta\eta)^2+(\Delta\phi)^2}}$ dependent on the lepton \pt\:
\begin{align}
\centering
  \Delta R &=\begin{cases}
    0.2, & \pt^{\text{lep}}\leq 50\,\GeV\\
    \frac{10\,\GeV}{\pt^{\text{lep}}}, & 50 \lt\ \pt^{\text{lep}} \le\ 200\,\GeV\\
    0.05, & \pt^{\text{lep}}\geq 200\,\GeV,
  \end{cases}\label{eq:miniIso}
\end{align}
where $\Delta \eta$ ($\Delta \phi$) is the distance in $\eta$ ($\phi$) between the directions of the lepton and the PF particle at 
the primary interaction vertex~\cite{TRK-11-001}. The average contribution of particles from additional \Pp\Pp\ interactions in the 
same or nearby bunch crossings (pileup) is estimated and subtracted from the $\pt^\text{sum}$ quantity. The isolation requirement is
$\pt^\text{sum} <  \min(5\GeV,\, 0.1 \cdot \pt^{\ell})$. Typical lepton identification and isolation efficiencies, measured in
samples of $\ensuremath{\cPZ/\Pgg^\star} \to \ell \ell$ events, are XX\% for electrons and YY\% for muons, with variations at the level of 
a few percent depending on \pt\ and $\eta$.

The PF particles are clustered to form jets using the anti-\kt clustering algorithm~\cite{antikt} with a distance parameter of 0.5, as 
implemented in the {\FASTJET} package~\cite{Cacciari:2011ma}. The contribution to the jet energy from pileup is estimated on an 
event-by-event basis using the jet area method described in \reference~\cite{cacciari-2008-659}, and is subtracted from the overall jet \pt.
Jets from pileup interactions are suppressed using a multivariate discriminant based on the multiplicity of objects clustered in the jet,
the jet shape, and the impact parameters of the charged tracks with respect to the primary interaction vertex. Jets overlapping with 
the selected leptons are not considered in the analysis.

The selected events are required to contain at least two jets with $\pt > 30\GeV$ and $\abs{\eta} < 2.4$. At least one of these jets must 
be consistent with containing the decay of a heavy-flavor hadron, as identified using the medium operating point of the combined secondary 
vertex (CSVv2) bottom quark (b~quark) tagging algorithm~\cite{ref:btag}. We refer to such jets as b-tagged jets. The efficiency of this algorithm 
for b~quark jets in the $\pt$ range 30--400\GeV varies between approximately 60 and 75\% for $\abs{\eta} < 2.4$. The nominal misidentification 
rate for light-quark or gluon jets is approximately 1\%~\cite{ref:btag} for the chosen working point. Corrections are applied to the energy 
measurements of jets to account for non-uniform detector response and  are propagated consistently as a correction to \ptvecmiss. 
The corrected \MET\ magnitude is required to exceed 250\GeV. To reduce effects of instrumentally induced \MET\, we require that the
azimuthal angle between \ptvecmiss and the closest of the two leading $E_{T}$ jets exceeeds 0.8 radians.

To suppress single-lepton backgrounds originating from semi-leptonic \ttbar, 
\wjets, and single top production processes, a requirement on the transverse mass of the lepton-neutrino system
$\MT = \sqrt{2 \pt^{\ell} \MET (1 - cos(\phi))}$ is imposed, where $\phi$ is the angle between the transverse momentum of the lepton and \MET. 
For the background processes containing a single leptonically-decaying $W$ boson, there is a kinematic endpoint $M_{T} < M_{W}$, which can be 
blurred by limited detector resolution and off-shell $W$ mass effects. In this analysis we require $M_{T} > 150\GeV$, which significantly reduces 
single-lepton backgrounds. 

The residual background is dominated by dilepton \ttbar and single top events, where one of the leptons is not reconstructed and the 
presence of the additional neutrino from the second leptonically-decaying $W$ boson allows the event to pass the $M_{T}$ requirement.
In order to reduce this background, we apply loose lepton reconstruction requirements to look for the presence of second electon or muon
with $\pt > 5\GeV$, and reject the event if such a lepton is found. Moreover, events are rejected if, in addition to the selected lepton,
they contain reconstructed hadronic tau decay with $\pt > 20\GeV$ or an isolated track with $\pt > 10\GeV$.


\subsection{Signal Region Definitions}
\label{sec:presel}
Kinematic properties and jet multiplicity in signal events depend on the decay mode of top squarks, as well as on the 
mass splittings between top squark, chargino and neutralino. To achieve better overall sensitivity of the analysis, we optimize
signal region selections separately for signal processes shown in~\ref{sgnProc} and for different mass splitting regions.

As a basis of the search strategy for Eq.~(1a) and (1b) models, we require presence of at least four jets. 
Events are then categorized based on the value of the \MTtW variables~\cite{Bai:2012gs}. The \MTtW variable attempts to reconstruct
the events under the assumption that it has originated in a $\ttbar\rightarrow\ell\ell$ process with one undetected lepton and  
helps to discriminate signal from the dominant dilepton \ttbar background. For large mass differences between the 
top squark and the neutralino, $\Delta$M, the $\MTtW>200\GeV$ requirement significantly reduces the background while preserving good 
signal efficiency. In contrast, for small $\Delta$M such a requirement would result in a loss of large fraction of the signal.
To preserve sensitivity to both high and low $\Delta$M values, we keep all events but divide them in two orthogonal search regions 
with $\MTtW>200\GeV$ and $\MTtW<200\GeV$.

In events with heavy top squarks and small neutralino masses there is a significant probability for the two quarks to merge into a single 
jet due to large boost of hadronically decaying $W$ boson. These events would fail four-jet requirement. To recover acceptance for such topology,
 we define additional search region with 3 jets. Since this region targets signal events with large $\Delta$M values, only events with 
$\MTtW>200\GeV$ are considered.

To increase the sensitivity for the mixed decay scenario model in Eq.~(1c) with nearly degenerate chargino and neutralino, 
a search region with exactly two jets is added. It was found that in these events with low jet multiplicity modified topness 
variable:
\begin{align} \label{eq:modtopness}
t_\mathrm{mod} = \ln(\min S)~\text{ with }~ S(\vec{p}_W, p_{\nu,z}) = \frac{(m_W^2-(p_\nu+p_\ell)^2)^2}{a_W^4} + 
\frac{(m_t^2 - (p_{b}+p_W)^2)^2}{a_t^4}
\end{align}
provides better dilepton \ttbar\ rejection than \MTtW. Here $a_W = 5\GeV$ and $a_t = 15\GeV$ are resolution parameters. 
Events are required to have $t_{mod}>6.4$.

Finally, events in each of the search regions described above are further classified into different signal regions 
based on the value of \MET. Overall, we end up with nine signal regions summarized in Table.~\ref{tab:SR}.
\begin{table}
\begin{center}
\topcaption{\label{tab:SR} Summary of the signal region definitions.}
%\tiny
%\setlength{\tabcolsep}{1pt}
%\vspace{-4pt}
\begin{tabular}{|r|r|r|rrr|}
\hline
$N_{Jets}$ & $M_\mathrm{T2}^W$ [GeV] & $t_\mathrm{mod}$ & \multicolumn{3}{c|}{$E_\mathrm{T}^\mathrm{miss}$ [GeV]} \\
\hline
$\geq4$ & $\leq 200$ & & $250$--$325$ & $>325$ & \\
$\geq4$ & $> 200$ & & $250$--$350$ & $350$--$450$ & $>450$ \\
\hline
$=3$ & $>200$ & & $250$--$350$ & $>350$ & \\
\hline
$=2$ & & $>6.4$ & $250$--$350$ & $>350$ & \\
\hline
\end{tabular}
\end{center}
\end{table}


\subsection{Signal and background simulation}
\label{sec:mc}

Signal samples of top squark pair production are generated with \MADGRAPH V5 Monte Carlo (MC) event 
generator interfaced with Pythia V8.1. Background samples of \ttbar\ and single top events are generated using \POWHEG 
with a top quark mass of $m_\cPqt=172.5$~\GeV, the CTEQ6M parton distribution functions (PDF), 
and the parton showering and fragmentation performed using Pythia V8.1.
Samples of \wjets, $\ttbar+\text{boson}$, diboson ($\PW\PW$, $\PW\cPZ$, and $\cPZ\cPZ$), and triboson,
events are generated with \MADGRAPH V5~\cite{Alwall:2011uj}.

For both signal and background events, pileup interactions are simulated with \PYTHIA\ and superimposed on the hard collisions,
using a pileup multiplicity distribution that reflects the luminosity profile of the analyzed data.
The CMS detector response is simulated using a \GEANTfour-based model~\cite{Geant}.
The simulated events are reconstructed and analyzed with the same software used to process the collision data.

The measured trigger efficiencies are used to weight the simulated events to account for the trigger requirement.
Small differences between the b~tagging efficiencies measured in data and simulation~\cite{ref:btag} are 
corrected using data-to-simulation scale factors to adjust the b~tagging probability in simulated events.
Lepton selection efficiencies (reconstruction, identification, and isolation) are found to be consistent between data and 
simulation.

